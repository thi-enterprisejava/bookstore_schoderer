\documentclass[12pt,a4paper]{book}
\usepackage[utf8]{inputenc}
\usepackage{amsmath}
\usepackage{amsfonts}
\usepackage{amssymb}
\usepackage{graphicx}
\usepackage[left=0.50cm, right=0.50cm, top=0.50cm, bottom=0.50cm]{geometry}
\author{Michael Schoderer}
\title{Architektur-Beschreibung: Bookstore}
\begin{document}
	\chapter{Bookstore}
	\section{Beschreibung}
	Die Java-EE Anwendung 'Bookstore' wurde für die Vorlesung Implementierung von Informationssystemen an der Technischen Hochschule Ingolstadt entwickelt und ermöglicht das Speichern und Verwalten von Dokumenten (PDF, MOBI, TXT, ...).
	\section{Architektur der Anwendung}
	\subsection{Übersicht}
	Bei der Konzeption und Entwicklung wurde die Anwendung in die im folgenden aufgelisteten Schichten unterteilt:
	\begin{itemize}
		\item \textbf{GUI} Die Oberfläche wurde mithilfe von JavaServer Faces (JSF) umgesetzt. Bei der Gestaltung der Oberfläche wurden Templates eingesetzt, damit ein einheitliches Bild der einzelnen Webseiten gewährleistet werden kann
		\item \textbf{Geschäftslogik} Die Geschäftslogik wurde unabhängig von der graphischen Oberfläche entwickelt, um eine Trennung nach dem MVC-Model zu erreichen
		\item \textbf{Datenbank} Die Datenbankschicht wurde durch mehrere Interfaces gekapselt und stellt die benötigten Funktionen bereit, um die Domain-Objekte zu in die jeweilige Datenbank zu persistieren		
		\end{itemize}
\end{document}